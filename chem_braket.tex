\documentclass[parskip=half, pagesize=auto, version=last]{scrartcl}

\usepackage{fixltx2e}
\usepackage{lmodern}
\usepackage[T1]{fontenc}
\usepackage{textcomp}
\usepackage{array}
\usepackage{braket}
\usepackage{chem_braket}
\usepackage{microtype}

\addtokomafont{title}{\rmfamily}

\title{The \textsf{chem-braket} package}
\subtitle{Macros for chemist's notation bra--ket $\left(\mid\right)$ notation}
\author{Shiv Upadhyay\\\texttt{shivnupadhyay@gmail.com}}
\date{23--Oct--2022}


\begin{document}

\maketitle

Commands defined are:

\begin{tabular}{@{}*{4}{>{\ttfamily\textbackslash}c<{\{~\}}}>{(}l<{~versions)}@{}}
  cbra & cket & cbraket & small \\
  CBra & CKet & CBraket & expanding
\end{tabular}

These function exactly like the \textsf{braket} package. 
The ``small versions'' use fixed-size brackets independent of their
contents, whereas the ``expanding versions'' make the brackets and 
vertical lines expand to envelop their contents (internally using 
the \verb+\left+ and \verb+\right+ commands). 
You should use the vertical bar character ``\verb+|+'' to input any extra vertical lines. 
In \verb+\Braket+ these vertical lines will expand to match the arguments.
E.\,g.,

\begingroup
\renewcommand*{\arraystretch}{1.7}
\setlength{\tabcolsep}{10pt}
\begin{tabular}{@{}>{\footnotesize}r>{$\displaystyle}l<{$}@{}}
  \verb+\CBraket{ \phi | \frac{\partial^2}{\partial t^2} | \psi }+ & \CBraket{ \phi | \frac{\partial^2}{\partial t^2} | \psi } \\
  \verb+\CBraket{ \mu\nu | \lambda\sigma }+ & \CBraket{ \mu\nu | \lambda\sigma } \\
\end{tabular}
\endgroup

This was recommended in the original package, so that is what I have done:
\begin{verbatim}
Because each definition is so small, it makes no sense to have a 
complicated generic version for many bracket styles.  Instead, 
you can just copy the definitions and change \verb+\langle+ or \verb+\rangle+,
to what you like.
\end{verbatim}

\end{document}
